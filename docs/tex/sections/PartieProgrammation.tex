\section{Partie Programmation} 
D'abord, notre programme doit être capable de communiquer avec le capteur ultrason afin de récupérer la distance et la vérifier. Pour cela, nous utilisons la bibliothèque grovepi que nous avons réussi à implémenter pendant les séances de TP et qui contient toutes les fonctions de haut niveau nécessaire pour lire et écrire des données via le bus I2C. Donc, nous avons inclus la bibliothèque dans le programme principale.\\
Après les initialisations, le programme entre dans une boucle infinie dans laquelle à chaque 0.7 secondes, il récupère la distance calculé par le capteur ultrason et la vérifie. Si le distance est supérieur à 1.5 mètres il continue à boucler, si c'est entre 1 mètre et 1.5 mètres il lance un appel système pour jouer le message d'avertissement.
Finalement, si la distance est inférieure à 1 mètre, le programme lance un premier appel système pour faire la capture photo et un second appel pour exécuter le script python qui s'occupe de l'envoi  de la photo prise par e-mail. 